\documentclass{beamer}
% \usepackage{polski}
% \usepackage[wust, pl]{collegeBeamer}
\usepackage[wust, en]{collegeBeamer}
\usepackage{lipsum}
\usepackage[style=numeric,sorting=none]{biblatex}
\addbibresource{mybibliography.bib}
\setbeamertemplate{bibliography item}{\insertbiblabel} % <--- number references
\setbeamertemplate{footline}[frame number]

\renewcommand{\arraystretch}{0.8} % Optional, adjusts row height if needed
\setlength{\tabcolsep}{4pt} % Optional, adjusts column spacing if needed
\usepackage{graphicx}
\AtBeginEnvironment{tabular}{\tiny} % Makes all table content tiny


\definecolor{hrefcol}{RGB}{0, 0, 255} % Example: blue color
\usepackage{algpseudocode}
\usepackage{algorithm}
% meta-data
\title{College Beamer\\ Presentation Themes}
\subtitle{Using \LaTeX\ to prepare slides}
\author{Justyna Witulska (\bhref{https://pwr.edu.pl/en/}{WUST})}
\date{Created: 16 March 2025}

% document body
\begin{document}

\maketitle

\begin{frame}
    This template is a secondary creation of 
    \bhref{https://www.overleaf.com/latex/templates/sintef-presentation/jhbhdffczpnx}{The Hong Kong Polytechnic University (PolyU) Beamer Presentation Theme} template from \bhref{mailto:qilong-kirov.liu@connect.polyu.hk}{Qi-long Liu} that is based on 
    \bhref{https://www.overleaf.com/latex/templates/sintef-presentation/jhbhdffczpnx}{SINTEF Presentation} template from \bhref{mailto:federico.zenith@sintef.no}{Federico Zenith}. \vspace{\baselineskip}

    This template is released under \bhref{https://creativecommons.org/licenses/by-nc/4.0/legalcode}{Creative Commons CC BY 4.0} license. \vspace{\baselineskip}

    All rights reserved by \bhref{mailto:federico.zenith@sintef.no}{Federico Zenith}.\vspace{\baselineskip}


    WUST logo source: \bhref{https://pwr.edu.pl/en/university/logo-and-presentations}{link (access: 16 March 2025)}
    
\end{frame}

\section{Introduction}

\begin{frame}{Example slide}
    \lipsum[1]
\end{frame}

\begin{frame}{An example of using mathematical formulas}
    The univariate random variable $Y$ is $\alpha$-stable distributed if its characteristic function $\varphi_Y(t):=E[exp(itY)]$ is defined in the following way \cite{nolan2021stable}:
    
    \begin{align}
    \varphi_Y(t) = \begin{cases}
    exp(-\gamma^{\alpha}|t|^{\alpha} \left( 1 - i\beta \cdot sgn(t) \cdot tan\left(\frac{\pi \alpha}{2}\right) \right) + i\delta t) &\text{  for  } \alpha \neq 1, \\
    exp(-\gamma|t| \left( 1 + i\beta \frac{2}{\pi} \cdot sgn(t) \cdot ln(|t|) \right) + i\delta t) &\text{  for  } \alpha = 1,
    \end{cases}
    \end{align}
    where $0<\alpha\leq 2$ is a stability index, $\beta \in [-1,1]$ is the skewness parameter, $\gamma>0$ is a scale parameter and $\delta \in \mathbb{R}$ is a location parameter. We denote $Y \sim S_{\alpha}(\gamma, \beta, \delta).$
\end{frame}

\begin{frame}{Code example}
    \begin{algorithm}[H]
    \begin{algorithmic}
    \caption{Name}
    \State $i \gets 11$
    \If{$i\geq 5$} 
    \State $i \gets i-1$
    \Else
    \If{$i\leq 3$}
        \State $i \gets i+2$
    \EndIf
    \EndIf 
    \end{algorithmic}
    \end{algorithm}
\end{frame}
    
\begin{frame}[fragile]{Figure example}
    \begin{figure}
        \centering
        \includegraphics[width=0.3\linewidth]{src/WUST/trans-logo.png}
        \caption{This is an example figure that you can download \bhref{https://pwr.edu.pl/en/university/logo-and-presentations}{here}.}
        \label{fig:enter-label}
    \end{figure}
\end{frame}


\section{Summary}
    \begin{frame}
    \frametitle{Conclusions}
    \begin{itemize}
    \item bullet 1
    \item bullet 2
    \item bullet 3
    \end{itemize}
\end{frame}

\renewcommand{\bibfont}{\normalsize}
\begin{frame}[allowframebreaks]{References}
    \printbibliography[heading=none] % <----- heading= none is added in order to prevent a duplicate heading
\end{frame}


    \QApage


\end{document}
